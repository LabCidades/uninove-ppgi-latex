\chapter{Introdução}
\label{ch:introducao}
	\begin{resumocapitulo}
		Este resumo pode ser utilizado para melhorar a comunicação com o leitor. As seções e subseções são configuradas de acordo com a norma ABNT adotada pela Uninove (tamanho da fonte, espaçamento...). As numerações de página estão alinhadas a direita no cabeçalho. Neste capítulo são mostrados exemplos para utilização de comandos de \textbf{citação}, \textbf{tabelas}, \textbf{quadros}, \textbf{equações} e \textbf{algoritmos}.
		%http://docs.uninove.br/arte/pdfs/Manual_de_Trabalhos_Academicos_ABNT_UNINOVE.pdf
	\end{resumocapitulo}

	\section{Comandos para citações diretas e indiretas}
	\label{sec:citacoes}
		\subsection{Citação Direta}
		\label{subsec:citacao_direta}
			Segundo \citeonline{mitchell1997machine}, aprendizagem de máquina ``[...] é como um programa de computador aprende pela experiência \textit{E}, com respeito a algum tipo de tarefa \textit{T} e performance \textit{P}, se sua performance \textit{P} nas tarefas em \textit{T}, na forma medida por \textit{P}, melhoram com a experiência''.

		\subsection{Citação Indireta}
		\label{subsec:citacao_indireta}
			O SCImago é um portal que fornece indicadores de produções científicas contidas no banco de dados do Scopus \cite{Villasenor-Almaraz2019}, sobre os principais periódicos do mundo \cite{DUggento2016}.
		
	\section{Montagem de Tabela}
	\label{sec:tabela}
		A seguir o exemplo de uma tabela, Tabela~\ref{tab:tab_identificador}. para auxiliar em tabelas mais complexas, está disponível a ferramenta \textbf{Tables Generator} (\href{https://www.tablesgenerator.com/}{https://www.tablesgenerator.com}).
		\begin{table}[!ht]
			\centering
				\caption{Descrição da tabela}
				\label{tab:tab_identificador}
				\begin{tabular*}{\columnwidth}{@{\extracolsep{\fill}}lrccc@{}}
					\toprule[1pt]{}\textbf{Desc. 1} & \textbf{Desc. 2} & \textbf{Desc. 4} & \textbf{Desc. 5} & \textbf{Desc. 6}\\\hline
					Item 1		& 901     	& 376  	& 4,738 & 21,317	\\
					Item 2		& 790		& 654  	& 5,913 & 45,540	\\
					Item 3 		& 333		& 215  	& 5,616 & 10,500	\\
					\bottomrule[1pt]
				\end{tabular*}
				\raggedright
				\amostra{2.024} \\% determina o tamanho de uma amostra
				\fontetabela{Autor} % alinha o nome do autor à esquerda
		\end{table}

	\section{Montagem de Quadro}
	\label{sec:quadro}
		\begin{quadros}[ht!]
			\caption{Descrição dos dados contidos no quadro.}
			\label{quad:contribuicoes_annals}
				\centering
				\begin{small}
					\def\arraystretch{1.1}
					\begin{tabular}{|p{1.0cm}|p{14.0cm}|}
						\hline
						\textbf{\#} & \textbf{Descrição} \\\hline
						1	& \textit{Linha 1} \\\hline
						2	& \textit{Linha 2} \\\hline
						3	& \textit{Linha 3} \\\hline
						4	& \textit{Linha 4} \\\hline
						5	& \textit{Linha 5} \\\hline
					\end{tabular}
				\end{small}
				\fonte{\cite{Abbasi2011}}
		\end{quadros}

	\section{Montagem de Equação}
	\label{sec:equacao}
		\begin{definicao}{Média aritmética}
			Para uma amostra $ X=\{x_1,, x_2, \ldots,x_n\} $ de observações, onde $ n $ é o número de observações, se define a média aritmética da seguinte forma:
			\begin{equation}
			\mu(X)=\dfrac{1}{n}\sum\limits_{x \in X}x
			\end{equation}
			\end{definicao}
			\begin{proposicao}
			Se $ k $ é uma constante então multiplicar a média de uma amostra $ X $ é o mesmo de multiplicar cada elemento de $ X $ por $ k $, isto é, $ k \times \mu(X) = \dfrac{1}{n} \sum\limits_{x \in X}x\times k $.
			\end{proposicao}
			\begin{prova}
			Desenvolve-se a igualdade:
			\begin{align*}
			k \times \mu(X) &= \dfrac{1}{n} \sum\limits_{x \in X}xk \\
			& \Longleftrightarrow  \dfrac{(x_1k,x_2k, \ldots, x_nk)}{n} \\
			& \Longleftrightarrow  \dfrac{nk \times (x_1,x_2, \ldots, x_n)}{n} \\
			& \Longleftrightarrow   k \times \dfrac{(x_1,x_2, \ldots, x_n)}{n} \\
			& \Longleftrightarrow   k \times \mu(X) \numberequation{1}
			\end{align*}
			\end{prova}
			Assim, concluí-se que $ k \times \mu(X) = \dfrac{1}{n} \sum\limits_{x \in X}x\times k $.

	\section{Montagem de Algoritmo}
	\label{sec:algortimo}
		Apresentação do Algoritmo~\ref{algorithm:algoritmo_descricao}.
		\begin{algorithm}
			\SetInd{0.5cm}{0.1cm}
			\Entrada{$Artigos$}
			\Saida{$Dataset$}
			\SetAlgoLined
			
			$ Dataset \leftarrow \emptyset $ \\
			\ForEach{$\text{artigo}~i \in \text{Artigos} $}{
				$ autor \leftarrow \emptyset $ \\
				\ForEach{$\text{autor}~k \in \text{artigo} $}{
					$ autor[k] \leftarrow \text{Extrair as informações de um dado~$i$ para o dado~$k$} $ \;
				}
				$ Dataset $ $\leftarrow \text{Adicionar os dados do}~dado $ \;
			}
			\caption{Texto que descreve o algoritmo.}
			\label{algorithm:algoritmo_descricao}
		\end{algorithm}

		\section{Inclusão de Figura}
		\label{sec:figura}
			A Figura~\ref{fig:identificador_da_figura} mostra os tipos de estruturação de dados.
			\begin{figure}[!ht]
				{\centering
					\caption{Descrição da figura.}
					\includegraphics[width=0.9\textwidth]{figuras/dados.png}
					\label{fig:identificador_da_figura}
					\fonte{Autor}
				} 		
			\end{figure}

		\obs{Isso é uma observação para conversar com o orientador ou para recordação}

		Utilize esse comando para tachar um texto, exemplo: \tachado{comprar}adquirir.
		\subsection{Subseção}
		\label{subsec:subseçao}
			Bla bla bla
